\pagestyle{plain}

\subsection{\textbf{目标追踪}}
\subsubsection{卡尔曼滤波}
\paragraph{卡尔曼滤波的基本原理}~{}

        卡尔曼滤波算法是指通过上一时刻的状态数据的估计值对当前的观测值进行修正和更新,采用“预测-实测-修正”的计算方式求得数据的最优预测值。在数据的处理过程中,它可以对数据中的噪声起到一定的抑制作用,实现对目标当前运动状态数据的平滑处理和将来运动状态的估计预测,数据的更新过程是依据目标的状态方程进行更新,主要包括对状态变量和量测数据的更新,状态变量X的更新方程为:
    
                 \begin{center}
                     $\begin{array}{r}\boldsymbol{X}(k)=\boldsymbol{F} \boldsymbol{X}(k-1)+\boldsymbol{G} V(k-1) \\P(k)=\boldsymbol{F} P(k-1) \boldsymbol{F}^{\mathrm{T}}+Q\end{array}$
                 \end{center}
                 
\begin{flushleft}
式中,$F$为状态矩阵;$Q$为数据噪声;$P(k)$为协方差。测量数值的更新方程为:
\end{flushleft}

                \begin{center}
                    $\begin{array}{c}K(k)=P(k) \boldsymbol{H}^{\mathrm{T}} \boldsymbol{H}\left(\boldsymbol{H} P(k) \boldsymbol{H}^{\mathrm{T}}+W\right)^{-1} \\\boldsymbol{X}(k+1)=\boldsymbol{X}(k)+k(Z(k)-\boldsymbol{X}(k) \boldsymbol{H}) \\P(k+1)=(\boldsymbol{I}-k \boldsymbol{H}) P(k)\end{array}$        
                \end{center}
                
\begin{flushleft}
    式中,$K(k)$ 为卡尔曼增益;$Z(k)$为观测数据;$\boldsymbol{H}$ 为观测矩阵;$W$为噪声; $\boldsymbol{I}$ 为单位矩阵。
\end{flushleft}
\paragraph{卡尔曼滤波的特性}~{}


    卡尔曼滤波具有如下的特性:
    \begin{enumerate}[(1)]
        \item 卡尔曼滤波算法不仅适用于平稳序列的滤波,而且适用于非平稳或平稳马尔可夫序列或高斯——马尔可夫序列的滤波,因此应用范围十分广泛。 
        \item 由于卡尔曼滤波的基本方程是时间域内的递推形式,其计算过程是一个不断地“预测-修正”过程,在求解时不要求存储大量的数据,并且一旦观测到了新的数据,随时可以算得新的滤波值,因此这种滤波方法非常便于实时处理。
        \item 增益矩阵与观测值无关,可以预先算出;增益矩阵与观测噪声方差阵$D_{\Delta}(k)$ 成反比,即当观测噪声增大时,滤波增益应取小一些,以减弱观测噪声对滤波值的影响;与系统动态噪声方差阵 $D_{\Omega}(k)$ 成正比。即当系统噪声变小时,增益矩阵应小些以便给与较小的修正。
        \item $D_{X}(k / k)$  是所有估计的最小方差阵。
    \end{enumerate}


\paragraph{卡尔曼滤波的不足}~{}

        卡尔曼滤波中的状态方程,常用数学模型来描述一个物理问题,这种陈述常受到对客观物理现象的了解和数学表达程度的局限而产生模型误差。观测方程的表达式要求是线性形式,而实际的观测量和状态参数间大都是非线性函数,非线性二次以上高次项的舍去,以及观测粗差等原因,也将使观测方程产生模型误差。
        
        理想情况下,卡尔曼滤波是线性无偏最小方差估计,根据滤波稳定性原理,对于一致完全可控和一致完全可观测系统,随着时间的推移,观测数据的增多,滤波估计的精度应该越来越高,滤波误差方差阵或者趋于稳态值,或者有界。但是在实际应用中,由滤波得到的状态估计可能是有偏的,且估计误差的方差也可能很大,远远超出了按计算公式计算的方差所定出的范围,更有甚者,其滤波误差的均值与方差都有可能趋于无穷大,这种现象,在滤波理论中称为滤波的发散现象。显然当滤波发散时,就完全失去了滤波的最优作用。因此在实际应用中,必须抑制这种现象。 
        
        导致最优滤波发散的原因主要有以下几种:
        \begin{enumerate}[(1)]
            \item 描述系统特性的数学模型和噪声模型的统计模型不准确,不能真实地反映物体过程,使模型与获得的观测值不匹配,导致滤波发散。
            \item 观测方程表达式是线性,而实际观测量与状态参数之间大都是非线性函数,非线性二次以上的高次项舍去,以及观测粗差等原因,导致模型存在误差,可能会导致发散。
            \item 卡尔曼滤波理论对动态系统提出了严格的要求,即要求系统噪声和观测噪声为零均值白噪声。这一条件在实践中难以满足,导致卡尔曼滤波产生发散现象,从而使状态估计不可信。可见动态噪声和观测噪声的方差和协方差估计是引起发散现象的重要因素。
            \item 卡尔曼滤波是递推过程,随着滤波步数的增加,舍入误差逐渐积累,如果计算机步长有限,这种积累有可能使估计的误差方差阵失去非负定性甚至失去对称性,使增益矩阵的计算值逐渐失去合适的加权作用而导致发散。
        \end{enumerate}
        
        应该指出,以上各种原因也不一定会引起滤波发散,视具体情况而定。

\subsubsection{粒子滤波}

\paragraph{地波雷达的多种探测方法}~{}

        高频地波雷达(High Freguency Surface Wave Radar, HFSWR)是大范围海上船只目标监视监测的主要手段,它利用高频电磁波(3~30MHz)沿海面爬行来实现超视距的目标(船只、低空飞机等)探测,可以实时提供运动目标的位置和航速航向等信息,探测距离最远可达300km,因此它又被称地波超视距雷达。
        
        传统的高频地波雷达目标探测方法采用的是先检测后跟踪(Detect-Before-Track,DBT)的思想,该方法对于回波能量较弱、信噪比较低的目标单时刻的检测效果不佳,从而使得连续时刻的目标跟踪性能下降。
        
        检测与跟踪联合处理方法可以解决弱目标检测困难的问题,其思路是:对单帧雷达回波数据不进行目标有无判断,而是利用目标在时空上的关联特性和杂波噪声的随机性,进行多帧数据累积,从而实现同一目标的回波能量累积,由此提高目标信噪比,完成目标的检测判决和航迹跟踪。检测与跟踪联合处理方法由于不设置检测门限,能充分利用目标回波谱的原始信息,即可减少先检测后跟踪过程中的点迹与航迹关联问题,亦可降低算法复杂度。
       
        目前,国内外已发展了多种检测前跟踪(Track-Before-Detect,TBD)方法,均可被用于实现地波雷达目标检测与跟踪联合处理。主要方法有:动态规划(Dynamic Programming,DP)法、粒子滤波(Particle Filter,PF)法和三维匹配滤波(3-D Matched Filters)法等。
        
\paragraph{ 粒子滤波法的优势}~{}

        粒子滤波法将目标航迹跟踪问题转换为目标的概率密度函数估计问题,相对于动态规划、三维匹配滤波等方法,它具有估计的目标状态在理论上最优的特点,适合于类似地波雷达这种非线性、非高斯的系统,且因其递归结构特点算法容易实现。
        
\paragraph{粒子滤波法原理}~{}

        在粒子滤波法中,通过使用序贯重要性采样方法对目标状态进行采样,得到一组带有权值的随机样本即“粒子” $\left\{x_{1: k}^{i}, w_{k}^{i}\right\}_{i=1}^{N}$ ,当采样的样本集足够大的时候,经过重要性采样获得的这组随机粒子就可以用来描述目标的后验概率密度函数   $p\left(x_{1: k} \mid Z_{1: k}\right)\left(x_{1: k}=\left\{x_{j} \mid j=1,2, \cdots, k\right\}\right.  $ 表示目标的状态序列,  $Z_{1: k}=\left\{z_{1}, z_{2}, \cdots\right. ,  \left.z_{k}\right\}  $ 表示目标的量测值序列, l 为似然比, k 为自然数)。一般在经过多次迭代后,粒子会出现退化现象,因此,在重要性采样结束后,还需要对粒子进行重采样。最终目标状态的后验概率密度可表示为 
      \begin{center}
           $p\left(x_{1, k} \mid Z_{1, k}\right) \approx \sum_{i=1}^{N} w_{k}^{i} \delta\left(x_{1, k}-x_{1, k}^{i}\right),$
      \end{center}
  \begin{flushleft}
    式中,$i$ 为自然数。
\end{flushleft}

        粒子滤波的粒子权重既可以构造检测似然比以实现目标检测,又可以通过粒子权重与粒子状态的加权估计目标的状态,从而实现目标的检测与跟踪联合处理。粒子权重与检测似然比之间的关系为
     \begin{center}
           $L_{k}=\frac{p\left(Z_{k} \mid H_{1}\right)}{p\left(Z_{k} \mid H_{0}\right)} \approx \prod_{\substack{i \in C_{i}\left(x_{k}\right) \\ j \in C_{j}\left(x_{k}\right)}} l\left(z_{k}^{(i, j)} \mid x_{k}\right)=\frac{1}{N} \sum_{i=1}^{N} w_{k}^{i},$
     \end{center}
 \begin{flushleft}
      式中,  $L_{k}$  为目标k时刻的似然比;  $H_{1}$  为目标存在;  $H_{0}$  为目标不存在;  $p\left(z_{k} \mid H_{1}\right)$  为目标存在时的概率密度;  $p\left(z_{k} \mid H_{0}\right)$  为目标不存在时的概率密度;  $C_{i}$  和  $C_{j}$  为受目标影响的单元; 则目标k时刻的递归似然比计算公式为  $\Lambda_{k} \approx \Lambda_{k-1} \prod_{\substack{i \in C_{i}\left(x_{k}\right) \\ j \in C_{j}\left(x_{k}\right)}} l\left(z_{k}^{(i, j)} \mid x_{k}\right) \approx \Lambda_{k-1} \frac{1}{N} \sum_{i=1}^{N} w_{k}^{i}$  。
 \end{flushleft}
 
        将计算得到的目标似然比与设定的阈值进行比较,判断目标是否存在。若目标存在,则在完成粒子重采样后按照公式 $x_{k}=\sum_{i=1}^{N} w_{k}^{i} \hat{x}_{k}^{i}$ 进行目标的状态估计。